\documentclass[11pt]{article}
\usepackage{fullpage}
\usepackage{verbatim}
\usepackage{amsmath}

\title{Stat 541 Homework \#1}
\author{Kenny Flagg}
\date{February 1, 2016}

\begin{document}
\maketitle

\begin{enumerate}

\item %1
~\vspace{-33pt}
\begin{align*}
\sum_{i=1}^a\sum_{j=1}^{n_i}\left(y_{ij}-\bar{y}_{..}\right)^2
&=\sum_{i=1}^a\sum_{j=1}^{n_i}\left(y_{ij}^2-2y_{ij}\bar{y}_{..}+\bar{y}_{..}^2\right)\\
&=\sum_{i=1}^a\left(\sum_{j=1}^{n_i}y_{ij}^2-2\left(\sum_{j=1}^{n_i}y_{ij}\right)\bar{y}_{..}+\sum_{j=1}^{n_i}\bar{y}_{..}^2\right)\\
&=\sum_{i=1}^a\left(\sum_{j=1}^{n_i}y_{ij}^2-2y_{i.}\bar{y}_{..}+n_i\bar{y}_{..}^2\right)\\
&=\sum_{i=1}^a\sum_{j=1}^{n_i}y_{ij}^2-2\left(\sum_{i=1}^ay_{i.}\right)\bar{y}_{..}+\left(\sum_{i=1}^an_i\right)\bar{y}_{..}^2\\
&=\sum_{i=1}^a\sum_{j=1}^{n_i}y_{ij}^2-2y_{..}\bar{y}_{..}+N\bar{y}_{..}^2\\
&=\sum_{i=1}^a\sum_{j=1}^{n_i}y_{ij}^2-2y_{..}\frac{{y}_{..}}{N}+N\left(\frac{{y}_{..}}{N}\right)^2\\
&=\sum_{i=1}^a\sum_{j=1}^{n_i}y_{ij}^2-2\frac{{y}_{..}^2}{N}+\frac{{y}_{..}^2}{N}\\
&=\sum_{i=1}^a\sum_{j=1}^{n_i}y_{ij}^2-\frac{{y}_{..}^2}{N}
\end{align*}

\pagebreak
\item %2

\begin{enumerate}

\item %a
Under \(H_0\) the test statistic is \(F_0\sim\mathrm{F}(5, 24)\) so for each \(\alpha\) we reject \(H_0\) if:
\begin{center}\begin{tabular}{|c|c|}
\hline
Signifiance Level & Rejection Region \\
\hline
\(\alpha=0.01\) &  \(F_0\geq F_{0.01}(5, 24)=3.90\) \\
\(\alpha=0.05\) & \(F_0\geq F_{0.05}(5, 24)=2.62\) \\
\(\alpha=0.10\) & \(F_0\geq F_{0.10}(5, 24)=2.10\) \\
\hline
\end{tabular}\end{center}

\item %b
First we need the overall total:
\begin{equation*}
y_{..}=\sum_i\sum_jy_{ij}=\sum_iy_{i.}=50+75+100+90+60+75=450
\end{equation*}
Next, find \(SS_T\) and \(SS_{Trt}\):
\begin{equation*}
SS_T=\sum_i\sum_j\left(y_{ij}-\bar{y}_{..}\right)^2=\sum_i\sum_jy_{ij}^2-\frac{y_{..}^2}{N}=7858-\frac{450^2}{30}=1108
\end{equation*}
\begin{equation*}
SS_{Trt}=\sum_i\sum_j\left(\bar{y}_{i.}-\bar{y}_{..}\right)^2=\sum_i\frac{y_{i.}^2}{n_i}-\frac{y_{..}^2}{N}
=\frac{50^2}{5}+\frac{75^2}{5}+\frac{100^2}{5}+\frac{90^2}{5}+\frac{60^2}{5}+\frac{75^2}{5}-\frac{450^2}{30}=340
\end{equation*}
Now we can fill in the ANOVA table:
\begin{center}\begin{tabular}{ccccc}
Source of &   Sum   &      &  Mean  &             \\
Variation & Squares & d.f. & Square & \(F\)-Ratio \\
\hline\hline
Treatment &   340   &   5  &   68   &\(F_0=2.125\)\\
Error     &   768   &  24  &   32   &             \\
\hline
Total     &  1108   &  29  &        &             \\
\hline\hline
\end{tabular}\end{center}

\item %c
We would reject \(H_0\): \(\mu_1=\mu_2=\mu_3=\mu_4=\mu_5=\mu_6\) for \(\alpha=0.10\) only.

\item %d
With all of the original observations, the treatment means were \(\bar{y}_{1.}=10\),
\(\bar{y}_{2.}=15\), \(\bar{y}_{3.}=20\), \(\bar{y}_{4.}=18\), \(\bar{y}_{5.}=12\),
and \(\bar{y}_{6.}=15\). Treatment 3 had the largest mean. With the one observation
removed, the new mean for treatment 3 is \(\bar{y}_{3.}=16\) which is close to the
means for treatments 2 and 6 and is no longer the largest mean. Thus there is less
between-treatment variability so \(MS_{Trt}\) will decrease.

\end{enumerate}

\pagebreak
\item %3
I ran the following SAS code:
%\verbatiminput{hw1.sas}
\begin{verbatim}
data catalysts;
  input catalyst concentration @@;
datalines;
1 58.2 1 57.2 1 58.4 1 55.8 1 54.9
2 56.3 2 54.5 2 57.0 2 55.3
3 50.1 3 54.2 3 55.4
4 52.9 4 49.9 4 50.0 4 51.7
;

proc glm data = catalysts;
  class catalyst;
  model concentration = catalyst;
\end{verbatim}

Here is the resulting ANOVA table:
%\verbatiminput{hw1.lis}
\begin{verbatim}
                               The GLM Procedure
 
                      Dependent Variable: concentration   

                                       Sum of
 Source                     DF        Squares    Mean Square   F Value   Pr > F

 Model                       3     85.6758333     28.5586111      9.92   0.0014

 Error                      12     34.5616667      2.8801389                   

 Corrected Total            15    120.2375000                                  
\end{verbatim}

\item %4

\begin{enumerate}

\item %a
Yes, we can compute:
\begin{equation*}
SS_E=\sum_i(n_i-1)s^2_i
\end{equation*}

\item %b
No, we need the sample variance \(s^2\) for the whole sample so we can compute:
\begin{equation*}
SS_{Trt}=\left(\sum_in_i-1\right)s^2
\end{equation*}

\item %c
No, we need \(s^2\) to find \(SS_{Trt}\) as above and then we can compute:
\begin{equation*}
SS_T=SS_{Trt}+SS_E
\end{equation*}

\end{enumerate}

\end{enumerate}

\end{document}
