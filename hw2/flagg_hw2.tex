\documentclass[11pt]{article}
\usepackage{fullpage}

\usepackage{amsmath}
\setcounter{MaxMatrixCols}{15}

\title{Stat 541 Homework \#2}
\author{Kenny Flagg}
\date{February 8, 2016}

\begin{document}
\maketitle

\begin{enumerate}

\item ~%1
\vspace{-22pt}\begin{center}\begin{tabular}{rrrr}
Circuit 1 & Circuit 2 & Circuit 3 & \\
\hline
 9 & 20 &  6 & \\
12 & 21 &  5 & \\
10 & 23 &  8 & \\
 8 & 17 & 16 & \\
15 & 30 &  7 & \\
\hline
\(y_{1.}=54\) & \(y_{2.}=111\) & \(y_{3.}=42\) & \(y_{..}=207\)
\end{tabular}\end{center}
\begin{enumerate}

\item %a
Normal equations:
\begin{align}
207&=15\widehat\mu+5\widehat\tau_1+5\widehat\tau_2+5\widehat\tau_3\\
 54&=\:\;5\widehat\mu+5\widehat\tau_1\\
111&=\:\;5\widehat\mu\qquad\:\:\,+5\widehat\tau_2\\
 42&=\:\;5\widehat\mu\qquad\qquad\quad\:+5\widehat\tau_3
\end{align}

\item %b
If \(\tau_1+\tau_2+\tau_3=0\), equation (1) becomes
\begin{equation*}
207=15\widehat\mu
\end{equation*}
so
\begin{equation*}
\widehat\mu=\frac{207}{15}=13.8
\end{equation*}

Equation (2) becomes
\begin{equation*}
54=5\times 13.8+\widehat\tau_1
\end{equation*}
so
\begin{equation*}
\widehat\tau_1=\frac{54-69}{5}=-3
\end{equation*}

Equation (3) becomes
\begin{equation*}
111=5\times 13.8+5\widehat\tau_2
\end{equation*}
so
\begin{equation*}
\widehat\tau_2=\frac{111-69}{5}=8.4
\end{equation*}

Finally, equation (4) becomes
\begin{equation*}
42=5\times 13.8+5\widehat\tau_3
\end{equation*}
so
\begin{equation*}
\widehat\tau_3=\frac{42-69}{5}=-5.4
\end{equation*}

\item %c
If \(\tau_2=0\), equation (3) becomes
\begin{equation*}
111=5\widehat\mu
\end{equation*}
so
\begin{equation*}
\widehat\mu=\frac{111}{5}=22.2
\end{equation*}

Equation (2) becomes
\begin{equation*}
54=5\times 22.2+\widehat\tau_1
\end{equation*}
so
\begin{equation*}
\widehat\tau_1=\frac{54-111}{5}=-11.4
\end{equation*}

And equation (4) becomes
\begin{equation*}
42=5\times 22.2+5\widehat\tau_3
\end{equation*}
so
\begin{equation*}
\widehat\tau_3=\frac{42-111}{5}=-13.8
\end{equation*}

\item %d
If \(\mu=5\), equation (2) becomes
\begin{equation*}
54=5\times 5+5\widehat\tau_1
\end{equation*}
so
\begin{equation*}
\widehat\tau_1=\frac{54-25}{5}=5.8
\end{equation*}

Equation (3) becomes
\begin{equation*}
111=5\times 5+\widehat\tau_2
\end{equation*}
so
\begin{equation*}
\widehat\tau_2=\frac{111-25}{5}=17.2
\end{equation*}

And equation (4) becomes
\begin{equation*}
42=5\times 5+5\widehat\tau_3
\end{equation*}
so
\begin{equation*}
\widehat\tau_3=\frac{42-25}{5}=3.4
\end{equation*}

\item ~%e
\vspace{-22pt}\begin{center}\begin{tabular}{ccc}
Constraint&\(\widehat\tau_1+\widehat\tau_2-2\widehat\tau_3\)&
\(\widehat\mu+\widehat\tau_1+\widehat\tau_3\)\\
\hline
(b)&\(-3+8.4-2\times(-5.4)=16.2\)&\(13.8-3-5.4=5.4\)\\
(c)&\(-11.4+0-2\times(-13.8)=16.2\)&\(22.2-11.4-13.8=-3\)\\
(d)&\(5.8+17.2-2\times(3.4)=16.2\)&\(5+5.8+3.4=14.2\)\\
\end{tabular}\end{center}

\item %f
The estimates of \(\widehat\mu+\widehat\tau_1+\widehat\tau_3\) differ because
\(\widehat\tau_3\) is not uniquely estimible. \(\widehat\mu+\widehat\tau_1\)
is uniquely estimible, but the value of \(\widehat\tau_3\) alone depends upon
the constraint used.

\end{enumerate}

\pagebreak
\item %2
\begin{enumerate}

\item ~%a
\vspace{-11pt}\begin{equation*}
X=\begin{bmatrix}
1&1&0\\
1&1&0\\
1&1&0\\
1&1&0\\
1&1&0\\
1&0&0\\
1&0&0\\
1&0&0\\
1&0&0\\
1&0&0\\
1&0&1\\
1&0&1\\
1&0&1\\
1&0&1\\
1&0&1
\end{bmatrix}\qquad
X'X=\begin{bmatrix}
15&5&5\\
5&5&0\\
5&0&5
\end{bmatrix}\qquad
(X'X)^{-1}=\frac{1}{5}\begin{bmatrix}
1&-1&-1\\
-1&2&1\\
-1&1&2\\
\end{bmatrix}
\end{equation*}
\begin{equation*}
X'y=\begin{bmatrix}
1&1&1&1&1&1&1&1&1&1&1&1&1&1&1\\
1&1&1&1&1&0&0&0&0&0&0&0&0&0&0\\
0&0&0&0&0&0&0&0&0&0&1&1&1&1&1
\end{bmatrix}\begin{bmatrix}
9\\12\\10\\8\\15\\20\\21\\23\\17\\30\\6\\5\\8\\16\\7
\end{bmatrix}\\
=\begin{bmatrix}
207\\54\\42
\end{bmatrix}
\end{equation*}

\item ~%b
\vspace{-11pt}\begin{equation*}
\theta=\begin{bmatrix}
\mu\\ \tau_1\\ \tau_3
\end{bmatrix}
\end{equation*}

\item ~%c
\vspace{-11pt}\begin{equation*}
\widehat\theta=\begin{bmatrix}
\widehat\mu\\ \widehat\tau_1\\ \widehat\tau_3
\end{bmatrix}=(X'X)^{-1}X'y=\frac{1}{5}\begin{bmatrix}
1&-1&-1\\
-1&2&1\\
-1&1&2\\
\end{bmatrix}\begin{bmatrix}
207\\54\\42
\end{bmatrix}
=\begin{bmatrix}
22.2\\-11.4\\-13.8
\end{bmatrix}
\end{equation*}

\pagebreak
\item ~%d
\vspace{-22pt}\begin{equation*}
X=\begin{bmatrix}
1&1&0\\
1&1&0\\
1&1&0\\
1&1&0\\
1&1&0\\
1&0&1\\
1&0&1\\
1&0&1\\
1&0&1\\
1&0&1\\
1&-1&-1\\
1&-1&-1\\
1&-1&-1\\
1&-1&-1\\
1&-1&-1
\end{bmatrix}\qquad
X'X=\begin{bmatrix}
15&0&0\\
0&10&5\\
0&5&10
\end{bmatrix}\qquad
(X'X)^{-1}=\frac{1}{15}\begin{bmatrix}
1&0&0\\
0&2&-1\\
0&-1&2\\
\end{bmatrix}
\end{equation*}
\begin{equation*}
X'y=\begin{bmatrix}
1&1&1&1&1&1&1&1&1&1&1&1&1&1&1\\
1&1&1&1&1&0&0&0&0&0&-1&-1&-1&-1&-1\\
0&0&0&0&0&1&1&1&1&1&-1&-1&-1&-1&-1
\end{bmatrix}\begin{bmatrix}
9\\12\\10\\8\\15\\20\\21\\23\\17\\30\\6\\5\\8\\16\\7
\end{bmatrix}\\
=\begin{bmatrix}
207\\12\\69
\end{bmatrix}
\end{equation*}

\item ~%e
\vspace{-22pt}\begin{equation*}
\theta=\begin{bmatrix}
\mu\\ \tau_1\\ \tau_2
\end{bmatrix}
\end{equation*}

\item ~%f
\vspace{-22pt}\begin{align*}
\widehat\theta&=\begin{bmatrix}
\widehat\mu\\ \widehat\tau_1\\ \widehat\tau_2
\end{bmatrix}=(X'X)^{-1}X'y=\frac{1}{15}\begin{bmatrix}
1&0&0\\
0&2&-1\\
0&-1&2\\
\end{bmatrix}\begin{bmatrix}
207\\12\\69
\end{bmatrix}
=\begin{bmatrix}
13.8\\-3\\8.4
\end{bmatrix}\\
\widehat\tau_3&=-\widehat\tau_1-\widehat\tau_2=3-8.4=-5.4
\end{align*}

\end{enumerate}

\pagebreak
\item %3
\begin{enumerate}

\item %a
The new data are:
\begin{center}\begin{tabular}{rrrr}
Circuit 1 & Circuit 2 & Circuit 3 & \\
\hline
 9 & 20 &  6 & \\
12 & 21 &  5 & \\
10 & 23 &  8 & \\
 8 & 17 & 16 & \\
15 & 30 &    & \\
\hline
\(y_{1.}=54\) & \(y_{2.}=111\) & \(y_{3.}=35\) & \(y_{..}=200\)
\end{tabular}\end{center}
The new normal equations are:
\begin{align}
200&=14\widehat\mu+5\widehat\tau_1+5\widehat\tau_2+4\widehat\tau_3\\
 54&=\:\;5\widehat\mu+5\widehat\tau_1\\
111&=\:\;5\widehat\mu\qquad\:\:\,+5\widehat\tau_2\\
 35&=\:\;4\widehat\mu\qquad\qquad\quad\:+4\widehat\tau_3
\end{align}

\item %b
If \(5\tau_1+5\tau_2+4\tau_3=0\), equation (5) becomes
\begin{equation*}
200=14\widehat\mu
\end{equation*}
so
\begin{equation*}
\widehat\mu=\frac{200}{14}=14.286
\end{equation*}

Equation (6) becomes
\begin{equation*}
54=5\times 14.286+\widehat\tau_1
\end{equation*}
so
\begin{equation*}
\widehat\tau_1=\frac{54-71.429}{5}=-3.486
\end{equation*}

Equation (7) becomes
\begin{equation*}
111=5\times 14.286+5\widehat\tau_2
\end{equation*}
so
\begin{equation*}
\widehat\tau_2=\frac{111-71.429}{5}=7.914
\end{equation*}

Lastly, equation (8) becomes
\begin{equation*}
35=4\times 14.286+4\widehat\tau_3
\end{equation*}
so
\begin{equation*}
\widehat\tau_3=\frac{35-57.143}{4}=-5.536
\end{equation*}

\end{enumerate}

\pagebreak
\setcounter{enumi}{4}
\item ~%5
\vspace{-33pt}\begin{align*}
\frac{-\sum_{i=1}^{a-1}n_i\left(\bar{y}_{i.}-\bar{y}_{..}\right)}{n_a}
&=\frac{n_a\left(\bar{y}_{a.}-\bar{y}_{..}\right)-\sum_{i=1}^{a}n_i\left(\bar{y}_{i.}-\bar{y}_{..}\right)}{n_a}\\
&=\left(\bar{y}_{a.}-\bar{y}_{..}\right)-\frac{\left(\sum_{i=1}^{a}n_i\bar{y}_{i.}-\sum_{i=1}^{a}n_i\bar{y}_{..}\right)}{n_a}\\
&=\bar{y}_{a.}-\bar{y}_{..}-\frac{\left(\sum_{i=1}^{a}y_{i.}-N\bar{y}_{..}\right)}{n_a}\\
&=\bar{y}_{a.}-\bar{y}_{..}-\frac{\left(y_{..}-y_{..}\right)}{n_a}\\
&=\bar{y}_{a.}-\bar{y}_{..}
\end{align*}

\end{enumerate}

\end{document}

