\documentclass[11pt]{article}
\usepackage{fullpage}
\usepackage{graphicx}
\usepackage{verbatim}
\usepackage{amsmath}

\title{Stat 541 Homework \#3}
\author{Kenny Flagg}
\date{February 1, 2016}

\begin{document}
\maketitle

\begin{enumerate}

\setcounter{enumi}{1}
\item %2

\begin{enumerate}

\item %a
\(H_0\): \(\tau_1=\tau_2=\tau_3=\tau_4\); All effects are equal.

\(H_a\): \(\tau_i\neq\tau_j\) for some \(i\neq j\); Some effects differ.

\item %b
SAS produced the following output. (See code in the appendix.)

\begin{verbatim}
                            PORTLAND CEMENT PROBLEM

                               The GLM Procedure
 
                        Dependent Variable: strength   

                                       Sum of
 Source                     DF        Squares    Mean Square   F Value   Pr > F

 Model                       3    489740.1875    163246.7292     12.73   0.0005

 Error                      12    153908.2500     12825.6875                   

 Corrected Total            15    643648.4375                                  


             R-Square     Coeff Var      Root MSE    strength Mean

             0.760881      3.862817      113.2506         2931.813


 Source                     DF    Type III SS    Mean Square   F Value   Pr > F

 technique                   3    489740.1875    163246.7292     12.73   0.0005
\end{verbatim}

Since the \verb|technique| row has \(\text{p-value}=0.0005\), we reject \(H_0\)
with \(\alpha=0.05\). There is evidence that the mean tensile strength is not
the same for all mixing techniques.

\pagebreak
\item %c
Fisher's LSD output from SAS:

\begin{verbatim}
                            PORTLAND CEMENT PROBLEM

                               The GLM Procedure
 
                           t Tests (LSD) for strength

    NOTE: This test controls the Type I comparisonwise error rate, not the 
                           experimentwise error rate.


                     Alpha                            0.05
                     Error Degrees of Freedom           12
                     Error Mean Square            12825.69
                     Critical Value of t           2.17881
                     Least Significant Difference   174.48


          Means with the same letter are not significantly different.
 
 
             t Grouping          Mean      N    technique

                      A       3156.25      4    2        
                                                         
                      B       2971.00      4    1        
                      B                                  
                      B       2933.75      4    3        
                                                         
                      C       2666.25      4    4        
\end{verbatim}

All pairs of means are significantly different except for the means of
techniques 1 and 3.

\item %d
Bonferroni's procedure would result in fewer significant differences
because the minimum significant difference would be larger than that
used by Fisher's procedure. In particular, techniques 1 and 2 may
no longer be significantly different because their difference
(185.25 lb/in\(^2\)) is only slightly higher than Fisher's least
significant difference.

\item %e
Since the measurements differ in the number of significant digits,
I would ask if the measurements were made by the same individual and
the same equipment. Different individuals and equipment would be
sources of dependency between observations that this analysis does not
account for.

\end{enumerate}

\pagebreak
\item %3

\begin{enumerate}

\item %a
\(H_0\): \(\sigma^2_\tau=0\); There is no variation in the mean calcuim concentration between batches.

\(H_a\): \(\sigma^2_\tau>0\); There is variation in the mean calcuim concentration between batches.

\item %b
ANOVA table from SAS:

\begin{verbatim}
                           BATCH VARIABILITY PROBLEM

                               The GLM Procedure
 
                         Dependent Variable: calcium   

                                       Sum of
 Source                     DF        Squares    Mean Square   F Value   Pr > F

 Model                       5     0.18932222     0.03786444     10.38   <.0001

 Error                      30     0.10943333     0.00364778                   

 Corrected Total            35     0.29875556                                  


              R-Square     Coeff Var      Root MSE    calcium Mean

              0.633703      13.38846      0.060397        0.451111


 Source                     DF    Type III SS    Mean Square   F Value   Pr > F

 batch                       5     0.18932222     0.03786444     10.38   <.0001
\end{verbatim}

\pagebreak
\item %c
The REML estimates from \verb|PROC VARCOMP| are \(\widehat\sigma^2_\tau=0.0057028\)
and \(\widehat\sigma^2=0.0036478\).

\begin{verbatim}
                           BATCH VARIABILITY PROBLEM

                    Variance Components Estimation Procedure

                            Class Level Information
 
                      Class         Levels    Values

                      batch              6    1 2 3 4 5 6 


                    Number of Observations Read          36
                    Number of Observations Used          36


                         Dependent Variable:    calcium


                                 REML Iterations
 
       Iteration          Objective         Var(batch)         Var(Error)

               0    -184.7778274868       0.0057027778       0.0036477778
               1    -184.7778274868       0.0057027778       0.0036477778


        Convergence criteria met.                                       


                                 REML Estimates
 
                            Variance
                            Component       Estimate

                            Var(batch)     0.0057028
                            Var(Error)     0.0036478


                   Asymptotic Covariance Matrix of Estimates
 
                                   Var(batch)      Var(Error)

                   Var(batch)      0.00001595      -1.4785E-7
                   Var(Error)      -1.4785E-7      8.87086E-7
\end{verbatim}

\item %d
Boxplot of calcium concentration by batch, produced by SAS:
\begin{center}
\includegraphics[width=0.5\linewidth]{BoxPlot.png}
\end{center}
All of the batches have somewhat similar spreads. Batches 1, 2, and 3 have similar
means. These are noticeably higher than the means of batches 4, 5, and 6,
which are themselves quite similar.

\item %e
We do not have enough information to assess the assumption of independence
between batches. The pattern in the boxplot could occur if one person or
device measured the first three batches and another person or device measured
the second three batches. We need to know what equipment and operator made each
measurement.

\end{enumerate}

%\pagebreak
\item %4
Let \(\mu_0\) be the mean for the sham treatment, and let \(\mu_i\)
be the mean for the \(i\) h/day treatment, \(i=1, 2, 3\).

Ignoring the sham treatement, orthogonal linear and quadratic contrasts for the
treatment means are:
\begin{align*}
\Gamma_L&=0\mu_0-1\mu_1+0\mu_2+1\mu_3\\
\Gamma_Q&=0\mu_0+1\mu_1-2\mu_2+1\mu_3
\end{align*}
The estimates of the treatment means are
\begin{equation*}
\widehat\mu_0=5.6725\qquad\widehat\mu_1=6.1655\qquad\widehat\mu_2=5.4780\qquad\widehat\mu_3=6.3505
\end{equation*}
so the estimates of the contrasts are
\begin{align*}
\widehat\Gamma_L&=0\times 5.6725-1\times 6.1655+0\times 5.4780+1\times 6.3505=0.185\\
\widehat\Gamma_Q&=0\times 5.6725+1\times 6.1655-2\times 5.4780+1\times 6.3505=1.560
\end{align*}

\end{enumerate}

\pagebreak
\subsection*{Appendix: SAS Code}

\subsubsection*{Problem 2}

\verbatiminput{hw3p2.sas}

\subsubsection*{Problem 3}

\verbatiminput{hw3p3.sas}

\end{document}
