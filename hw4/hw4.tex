\documentclass[11pt]{article}
\usepackage{fullpage}
\usepackage{graphicx}
\usepackage{verbatim}
\usepackage{amsmath}

\title{Stat 541 Homework \#4}
\author{Kenny Flagg}
\date{February 24, 2016}

\begin{document}
\maketitle

\begin{enumerate}

\item %1

\begin{enumerate}

\item %a
\textbf{ANOVA on Untransformed Data}

\(H_0\): \(\mu^2_1=\mu^2_2=\mu^2_3=\mu^2_4=\mu^2_5\); The mean count per second
is the same for each treatment.

\(H_a\): \(\mu^2_i=\mu^2_j\) for some \(i\neq j\); Some treatment means are not
equal.

\begin{verbatim}
                     ANOVA RESULTS: No Transformation

                             The GLM Procedure
 
                      Dependent Variable: countsec   

                                     Sum of
 Source                    DF       Squares   Mean Square  F Value  Pr > F

 Model                      4    0.04752280    0.01188070    95.53  <.0001

 Error                     25    0.00310920    0.00012437                 

 Corrected Total           29    0.05063200                               


           R-Square     Coeff Var      Root MSE    countsec Mean

           0.938592      22.30408      0.011152         0.050000


 Source                    DF   Type III SS   Mean Square  F Value  Pr > F

 pCig                       4    0.04752280    0.01188070    95.53  <.0001
\end{verbatim}

The \verb|pCig| row has \(\text{p-value}<0.0001\) so there is strong evidence
that the treatments do not all have the same mean count per second.

\pagebreak
\item %b
\textbf{Tests of Contrasts}

\begin{verbatim}
                     ANOVA RESULTS: No Transformation

                             The GLM Procedure
 
                      Dependent Variable: countsec   

                                            Standard
Parameter                   Estimate           Error    t Value    Pr > |t|

Linear                    0.26700000      0.01410634      18.93      <.0001
Quadratic                 0.00540000      0.01727667       0.31      0.7572
Cubic                     0.00800000      0.01537203       0.52      0.6073
Quartic                   0.01040000      0.04157782       0.25      0.8045
\end{verbatim}

The linear contrast has \(\text{p-value}<0.0001\), so there is strong evidence
of a linear trend in the treatment mean count per second. The other contrasts
all have \(\text{p-value}>0.05\) so there is little to no evidence of quadratic,
cubic, or quartic trends.

\item %c
\textbf{Boxplots}

\begin{center}\includegraphics[width=0.5\linewidth]{orig9}\end{center}

The results from the tests of the contrasts agree with the boxplots, which show
a very linear trend with no curvature.

\pagebreak
\item %d
\textbf{Levene's Test}

\(H_0\): \(\sigma^2_1=\sigma^2_2=\sigma^2_3=\sigma^2_4=\sigma^2_5\); The
variances are equal for each treatment.

\(H_a\): \(\sigma^2_i=\sigma^2_j\) for some \(i\neq j\); Some treatment
variances are not equal.

\begin{verbatim}
                     ANOVA RESULTS: No Transformation

                             The GLM Procedure

            Levene's Test for Homogeneity of countsec Variance
               ANOVA of Absolute Deviations from Group Means
 
                             Sum of        Mean
       Source        DF     Squares      Square    F Value    Pr > F

       pCig           4    0.000752    0.000188       5.58    0.0024
       Error         25    0.000842    0.000034                     
\end{verbatim}

Levene's test gives a \(\text{p-value}=0.0024\) so there is strong evidence
that the variances differ between treatments.

\item %e
\textbf{Diagnostic Plots}

\includegraphics[width=0.5\linewidth]{orig}\includegraphics[width=0.5\linewidth]{orig3}

The residuals vs predicted values plot shows a clear fanning pattern of
increasing variance with increasing counts per second. The normal probability
plot does not show any serious deviation from normality aside from some extreme
values in the tails. It is appropriate to use Levene's test to check the
homogeneity of variance assumption.

\pagebreak
\item %f
\textbf{ANOVA on Square-Root Transformed Response}

\(H_0\): \(\mu^2_1=\mu^2_2=\mu^2_3=\mu^2_4=\mu^2_5\); The mean square root
count per second is the same for each treatment.

\(H_a\): \(\mu^2_i=\mu^2_j\) for some \(i\neq j\); Some treatment means are not
equal.

\begin{verbatim}
                 ANOVA RESULTS: Square Root Transformation

                             The GLM Procedure
 
                      Dependent Variable: sqrtcnt   

                                     Sum of
 Source                    DF       Squares   Mean Square  F Value  Pr > F

 Model                      4    0.30234533    0.07558633   164.99  <.0001

 Error                     25    0.01145339    0.00045814                 

 Corrected Total           29    0.31379872                               


           R-Square     Coeff Var      Root MSE    sqrtcnt Mean

           0.963501      10.76412      0.021404        0.198847


 Source                    DF   Type III SS   Mean Square  F Value  Pr > F

 pCig                       4    0.30234533    0.07558633   164.99  <.0001
\end{verbatim}

The \verb|pCig| row has \(\text{p-value}<0.0001\) so there is strong evidence
that the treatments do not all have the same mean square root count per second.

\pagebreak
\textbf{Tests of Contrasts}

\begin{verbatim}
                 ANOVA RESULTS: Square Root Transformation

                             The GLM Procedure
 
                      Dependent Variable: sqrtcnt   

                                            Standard
Parameter                   Estimate           Error    t Value    Pr > |t|

Linear                    0.63300870      0.02707429      23.38      <.0001
Quadratic                -0.13182571      0.03315909      -3.98      0.0005
Cubic                     0.05075728      0.02950352       1.72      0.0977
Quartic                  -0.00612794      0.07980028      -0.08      0.9394
\end{verbatim}

The linear contrast has \(\text{p-value}<0.0001\) and the quadratic contrast
has \(\text{p-value}=0.0005\), so there is strong evidence of linear and
quadratic trends in the mean square root count per second. The cubic and
quartic contrasts have \(\text{p-value}>0.05\) so there is little to no
evidence of cubic or quartic trends in the transformed data.

\textbf{Boxplots}

\begin{center}\includegraphics[width=0.5\linewidth]{sqrt9}\end{center}

The boxplots show an increasing trend with some quadratic curvature, but no
higher-order curvature is apparent. This agrees with the results from testing
the contrasts.

\pagebreak
\textbf{Levene's Test}

\(H_0\): \(\sigma^2_1=\sigma^2_2=\sigma^2_3=\sigma^2_4=\sigma^2_5\); The
variances are equal for each treatment.

\(H_a\): \(\sigma^2_i=\sigma^2_j\) for some \(i\neq j\); Some treatment
variances are not equal.

\begin{verbatim}
                 ANOVA RESULTS: Square Root Transformation

                             The GLM Procedure

             Levene's Test for Homogeneity of sqrtcnt Variance
               ANOVA of Absolute Deviations from Group Means
 
                             Sum of        Mean
       Source        DF     Squares      Square    F Value    Pr > F

       pCig           4    0.000563    0.000141       1.10    0.3773
       Error         25     0.00319    0.000128                     
\end{verbatim}

Levene's test on the transformed data gives \(\text{p-value}=0.3773\). There is
little to no evidence that that variances differ between the treatments. The
transformation has stabilized the variance.

\textbf{Diagnostic Plots}

\includegraphics[width=0.5\linewidth]{sqrt}\includegraphics[width=0.5\linewidth]{sqrt3}

The residuals vs predicted values plot shows very similar spreads for each
predicted value, so the variance can be assumed constant. The normal
probability plot shows no problems.

\pagebreak
\item %g
\textbf{Box-Cox Transformation}

\begin{center}\includegraphics[width=0.5\linewidth]{boxcox}\end{center}

The Box-Cox method recommends \(\lambda=0.34\), or approximately
\(y^*=\sqrt[3]{y}\).

\item %h
\textbf{Recommendation}

I recommend using the square root transformation. Levene's test on the original
data gives evidence (\(\text{p-value}=0.0024\)) the the homogeneity of variance
assumption does not hold, so ANOVA results on the original data are unreliable.
The square root transformation is reasonable because 0.5 is in the 95\%
confidence interval for \(\lambda\) from the Box-Cox procedure.

\end{enumerate}

\item %2

\begin{enumerate}

\item %a

The boxplot shows vastly different spreads among the groups, so yes it was
reasonable to do a WLS ANOVA using \(\displaystyle w_i=\frac{1}{s^2_i}\)
as the weights.

\item %b

The spread of a group does not appear to be related to the mean of the
group, so the Box-Cox procedure is unlikely to successfully stabilize
the variance.

\end{enumerate}

\pagebreak
\item %3
Fit the regression model \(\log(s_i)=\log(\theta)+\alpha\log(\bar{y}_i)+\epsilon_i\)

\begin{verbatim}
                ANOVA TO FIND EMPIRICAL SELECTION OF ALPHA

                             The GLM Procedure
 
                       Dependent Variable: logstd   

                                       Standard
     Parameter         Estimate           Error    t Value    Pr > |t|

     Intercept      10.83673889      1.10964799       9.77      0.0103
     logmean        -0.76145545      0.14854676      -5.13      0.0360
\end{verbatim}

\begin{center}\includegraphics[width=0.5\linewidth]{empirical}\end{center}

The least squares estimate of the slope is \(\widehat{\alpha}=-0.7614\) so the
suggested transformation is \(y^*=y^{1.7614}\)

\end{enumerate}

\pagebreak
\subsection*{Appendix: SAS Code}

\subsubsection*{Problem 1}

\verbatiminput{hw4p1.sas}

\subsubsection*{Problem 3}

\verbatiminput{hw4p3.sas}

\end{document}
