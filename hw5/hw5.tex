\documentclass[11pt]{article}
\usepackage{fullpage}
\usepackage{graphicx}
\usepackage{verbatim}
\usepackage{amsmath}

\title{Stat 541 Homework \#5}
\author{Kenny Flagg}
\date{March 4, 2016}

\begin{document}
\maketitle

\begin{enumerate}

\item %1
\emph{Problem 3.42 (a), sample size to detect a difference in means of 10 with power 0.90.}

\begin{enumerate}

\item %a
If \(\sigma=2\), a total of 9 observations (3 batteries of each brand) are
needed to detect a difference in mean battery life of 10 hours with a power
of at least 0.90. The actual power achieved is 0.998.

\item %b
If \(\sigma=3\), a total of 9 observations (3 batteries of each brand) are
needed to detect a difference in mean battery life of 10 hours with a power
of at least 0.90. The actual power achieved is 0.902.

\end{enumerate}

{\footnotesize\verbatiminput{hw5p1.lst}}

\pagebreak
\item %2
\emph{Problem 3.42 (a) again, sample size to detect a difference in means of 10 with power 0.95.}

\begin{enumerate}

\item %a
If \(\sigma=2\), a total of 9 observations (3 batteries of each brand) are
needed to detect a difference in mean battery life of 10 hours with a power
of at least 0.95. The actual power achieved is 0.998.

\item %b
If \(\sigma=2\), a total of 12 observations (4 batteries of each brand) are
needed to detect a difference in mean battery life of 10 hours with a power
of at least 0.95. The actual power achieved is 0.987.

\end{enumerate}

{\footnotesize\verbatiminput{hw5p2.lst}}

\item %3
\emph{In Problem 3.26, was random assignment used?}

Random assignment was not used. The treatment is brand, and it would not be
possible to randomly assign batteries to brands.

\pagebreak
\item %4
\emph{Orthogonal polynomial contrasts for 2mg, 4mg, 8mg, and 10mg.}

The mean is \(\bar{x}=6\), so the initial vectors are:
\begin{equation*}
\mathbf{v}_0=\begin{bmatrix}1 \\ 1 \\ 1 \\ 1\end{bmatrix}\qquad
\mathbf{v_1}=\begin{bmatrix}-4 \\ -2 \\ 2 \\ 4\end{bmatrix}\qquad
\mathbf{v}_2=\begin{bmatrix}16 \\ 4 \\ 4 \\ 16\end{bmatrix}\qquad
\mathbf{v}_3=\begin{bmatrix}-64 \\ -8 \\ 8 \\ 64\end{bmatrix}\qquad
\end{equation*}
\begin{equation*}
|\mathbf{v}_0|=\sqrt{1+1+1+1}=2\qquad\implies\qquad
\mathbf{u}_0=\frac{\mathbf{v}_0}{|\mathbf{v}_0|}=\frac{1}{2}\begin{bmatrix}1 \\ 1 \\ 1 \\ 1\end{bmatrix}
\end{equation*}

Linear Contrast Vector:
\begin{equation*}
\mathbf{v}_1\cdot\mathbf{u}_0=\frac{1}{2}(-4-2+2+4)=0
\end{equation*}
\begin{equation*}
\mathbf{w}_1=\mathbf{v}_1-(\mathbf{v}_1\cdot\mathbf{u}_0)\mathbf{u}_0=\mathbf{v}_1-0\mathbf{u}_0=\begin{bmatrix}-4 \\ -2 \\ 2 \\ 4\end{bmatrix}
\end{equation*}
\begin{equation*}
|\mathbf{w}_1|=\sqrt{16+4+4+16}=2\sqrt{10}\qquad\implies\qquad
\mathbf{u}_1=\frac{\mathbf{w}_1}{|\mathbf{w}_1|}=\frac{1}{\sqrt{10}}\begin{bmatrix}-2 \\ -1 \\ 1 \\ 2\end{bmatrix}
\end{equation*}

Quadratic Contrast Vector:
\begin{equation*}
\mathbf{v}_2\cdot\mathbf{u}_0=\frac{1}{2}(16+4+4+16)=20
\end{equation*}
\begin{equation*}
\mathbf{v}_2\cdot\mathbf{u}_1=\frac{1}{\sqrt{10}}(-32-4+4+32)=0
\end{equation*}
\begin{equation*}
\mathbf{w}_2=\mathbf{v}_2-(\mathbf{v}_2\cdot\mathbf{u}_0)\mathbf{u}_0-(\mathbf{v}_2\cdot\mathbf{u}_1)\mathbf{u}_1
=\mathbf{v}_2-20\mathbf{u}_0-0\mathbf{u}_1
=\begin{bmatrix}16 \\ 4 \\ 4 \\ 16\end{bmatrix}-\begin{bmatrix}10 \\ 10 \\ 10 \\ 10\end{bmatrix}
=\begin{bmatrix}6 \\ -6 \\ -6 \\ 6\end{bmatrix}
\end{equation*}
\begin{equation*}
|\mathbf{w}_2|=\sqrt{36+36+36+36}=12\qquad\implies\qquad
\mathbf{u}_2=\frac{\mathbf{w}_2}{|\mathbf{w}_2|}=\frac{1}{2}\begin{bmatrix}1 \\ -1 \\ -1 \\ 1\end{bmatrix}
\end{equation*}

Cubic Contrast Vector:
\begin{equation*}
\mathbf{v}_3\cdot\mathbf{u}_0=\frac{1}{2}(-64-8+8+64)=0
\end{equation*}
\begin{equation*}
\mathbf{v}_3\cdot\mathbf{u}_1=\frac{1}{\sqrt{10}}(128+8+8+128)=\frac{272}{\sqrt{10}}
\end{equation*}
\begin{equation*}
\mathbf{v}_3\cdot\mathbf{u}_2=\frac{1}{2}(-64+8-8+64)=0
\end{equation*}
\begin{align*}
\mathbf{w}_3=\mathbf{v}_3-(\mathbf{v}_3\cdot\mathbf{u}_0)\mathbf{u}_0-(\mathbf{v}_3\cdot\mathbf{u}_1)\mathbf{u}_1-(\mathbf{v}_3\cdot\mathbf{u}_2)\mathbf{u}_2
&=\mathbf{v}_3-0\mathbf{u}_0-\frac{288}{\sqrt{10}}\mathbf{u}_1-0\mathbf{u}_2\\
&=\frac{1}{5}\begin{bmatrix}-320 \\ -40 \\ 40 \\ 320\end{bmatrix}-\frac{1}{5}\begin{bmatrix}-272 \\ -136 \\ 136 \\ 272\end{bmatrix}
=\frac{1}{5}\begin{bmatrix}-48 \\ 96 \\ -96 \\ 48\end{bmatrix}
\end{align*}
\begin{equation*}
|\mathbf{w}_3|=\frac{48}{5}\sqrt{1+4+4+1}=\frac{48\sqrt{10}}{5}\qquad\implies\qquad
\mathbf{u}_3=\frac{\mathbf{w}_3}{|\mathbf{w}_3|}=\frac{1}{48\sqrt{10}}\begin{bmatrix}-1 \\ 2 \\ -2 \\ 1\end{bmatrix}
\end{equation*}

So the contrasts are:
\begin{align*}
\Gamma_L&=-2\mu_1-\mu_2+\mu_3+2\mu_4\\
\Gamma_Q&=\mu_1-\mu_2-\mu_3+\mu_4\\
\Gamma_C&=-1\mu_1+2\mu_2-2\mu_3+\mu_4\\
\end{align*}

%\pagebreak
\item %5
\emph{Cell count experiment with runs as blocks.}

\begin{enumerate}

\item %a
\emph{Model and parameters.}

The model is
\begin{equation*}
y_{ij}=\mu+\tau_i+\beta_j+\epsilon_{ij};\quad\epsilon_{ij}\overset{iid}{\sim}\mathrm{N}(0,\sigma^2)
\end{equation*}
where \(\mu\) is the mean cell count for the control group in experiment 6;
\(\tau_i\), \(i=2,3\), are the effects of the drugs on mean cell count; and
\(\beta_j\), \(j=1,\dots,5\), are the effects of the experimental runs on the
mean cell count.

\item %b
\emph{ANOVA hypotheses.}

\(H_0\): \(\tau_2=\tau_3=0\); Both drugs have no effect on mean cell count.

\(H_a\): \(\tau_i\neq 0\) for some \(i\); At least one drug has an effect on
mean cell count.

\pagebreak
\item %c
\emph{ANOVA results.}

There is strong evidence (\(F_0=6.78\), \(\text{p-value}=0.0138\)) that at
least one of the drugs has an effect on mean cell count.

{\footnotesize\verbatiminput{hw5p5c.lst}}

\item %d
\emph{Did either drug increase cell counts?}

Drug 1 has an estimated effect of \(\widehat{\tau}_2=147.83\) (\(t_0=3.14\),
\(\text{right-tailed p-value}=0.0052\)) so there is strong evidence that
drug 1 increases the mean cell count.

Drug 2 has an estimated effect of \(\widehat{\tau}_3=-4.17\) (\(t_0=-0.09\),
\(\text{two-tailed p-value}=0.9311\)) so there is no evidence that drug 2
has an effect on the mean cell count.

\item %e
\emph{Is a square root transformation appropriate?}

The square root transformation is not necessary. The studentized residuals
by predicted values plot does not show any serious violation of the constant
variance assumption.

\begin{center}
\includegraphics[width=0.4\textwidth]{hw5p5c1.png}
\end{center}

\item %f
\emph{ANOVA ignoring blocks.}

There is moderate evidence (\(F_0=3.37\), \(\text{p-value}=0.0620\)) that at
least one of the drugs has an effect on mean cell count.

{\footnotesize\verbatiminput{hw5p5f.lst}}

\item %g
\emph{Did blocking improve the analysis?}

Yes, blocking improved the analysis. In both analyses, the estimated drug
effects are the same, but their standard errors differ. Including the
blocks accounts for run-to-run variability so the estimates are more precise
and we can be more certain in our conclusion that drug 1 is effective.

\end{enumerate}

\item %6
\emph{Power analysis for Problem 3.10.}

\begin{center}\begin{tabular}{c|rrrrr|r}
Cotton Weight Percent & \multicolumn{5}{c|}{Tensile Strengths} & Mean \\
\hline
15 &  7 &  7 & 15 & 11 &  9 &  9.8 \\
20 & 12 & 17 & 12 & 18 & 18 & 15.4 \\
25 & 14 & 19 & 19 & 18 & 18 & 17.6 \\
30 & 19 & 25 & 22 & 19 & 23 & 21.6 \\
35 &  7 & 10 & 11 & 15 & 11 & 10.8
\end{tabular}

\(\widehat\sigma=\sqrt{MSE}=2.84\)
\end{center}

\begin{enumerate}

\item %a

A total of 70 observations (14 observations of each cotton weight percentage)
are needed to detect a linear trend in mean tensile strength with a power of
at least 0.90. The actual power achieved is 0.920.

\item %b

A total of 10 observations (2 observations of each cotton weight percentage)
are needed to detect a quadratic trend in mean tensile strength with a power
of at least 0.90. The actual power achieved is 0.904.

\end{enumerate}

{\footnotesize\verbatiminput{hw5p6.lst}}

\end{enumerate}

\pagebreak
\subsection*{Appendix: SAS Code}

\subsubsection*{Problems 1 and 2}

{\footnotesize\verbatiminput{hw5p1.sas}}

\subsubsection*{Problem 5}

{\footnotesize\verbatiminput{hw5p5.sas}}

\subsubsection*{Problem 6}

{\footnotesize\verbatiminput{hw5p6.sas}}

\end{document}
