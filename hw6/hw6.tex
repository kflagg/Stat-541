\documentclass[11pt]{article}
\usepackage{fullpage}
\usepackage{graphicx}
\usepackage{verbatim}
\usepackage{amsmath}

\usepackage{listings}
\lstloadlanguages{SAS}
\lstset{basicstyle=\footnotesize\ttfamily,columns=fixed,showstringspaces=false,
  showspaces=false,showtabs=false}

\usepackage{enumitem}
\setlist{parsep=5.5pt}

\usepackage{fancyhdr}
\pagestyle{fancy}
\lhead{Stat 541 Homework \#6}
\chead{April 4, 2016}
\rhead{Kenny Flagg}
\setlength{\headheight}{18pt}

\title{Stat 541 Homework \#6}
\author{Kenny Flagg}
\date{April 4, 2016}

\begin{document}
\thispagestyle{plain}
\maketitle

\begin{enumerate}

\item %1
{\it BIBD for wool cleaning agents.}

\begin{enumerate}

\item %a
{\it Verify that this study has the appropriate structure for a BIBD.}

There are \(a=4\) treatments, \(b=16\) blocks, \(k=3\) treatments applied to
each block. Each treatment appears in \(r=12\) blocks and each pair of
treatments appears in \(\lambda=8\) blocks. Then
\begin{equation*}
\frac{r(k-1)}{a-1}=\frac{(12)(2)}{3}=8=\lambda
\end{equation*}
so this a balanced incomplete block design.

\item %b
{\it State the model and the hypotheses to be tested.}

The model is
\begin{equation*}
y_{ij}=\mu+\tau_i+\beta_j+\epsilon_{ij};\qquad
\epsilon_{ij}\overset{iid}{\sim}\mathrm{N}(0,\sigma^2)
\end{equation*}
where \(i=1,2,3,4\) indexes treatments A, B, C, D, and \(j=1,2,\dots,16\)
indexes the batches.

The hypotheses being tested are
\begin{align*}
H_0\text{: }&\tau_1=\tau_2=\tau_3=\tau_4\text{,}\\
H_1\text{: }&\tau_i\neq\tau_{i'}\text{ for some }i\neq i'\text{.}
\end{align*}

The ANOVA table produced by SAS apears on the next page. The test statistic
is \(F=8.80\) with \(\text{p-value}=0.0003\), so at \(\alpha=0.05\) there is
enough evidence to conclude that there are differences in mean weight loss
among the different cleaning agents.

\pagebreak
\lstinputlisting[firstline=22,lastline=46]{hw6p1.lst}

\item %c
{\it Perform Bonferroni's multiple comparison test using both the sample means
and the least squares means. For each case, state which pairs of means are
considered statistically significantly different.}

Since there is evidence that the treatment means are not all equal, we can
perform Bonferroni's multiple comparisons procedure to test the hypotheses
\begin{align*}
H_0\text{: }&\mu_i=\mu_{i'}\text{,}\\
H_1\text{: }&\mu_i\neq\mu_{i'}
\end{align*}
for each pair, \(i\neq i'\).

SAS output appears on the next page. When applied to the sample means, at
\(\alpha=0.05\), we conclude that the mean of B is significantly different
from the means of A, C, and D. We have little evidence that there are
differences among the means of any of A, C, and D.

At \(\alpha=0.05\), the Bonferroni-adjusted p-values for comparisons of the
least squares means indicate significant differences between  A and C, A and D,
B and C, and B and D.

\pagebreak

\lstinputlisting[firstline=50,lastline=78]{hw6p1.lst}
\vfill
\lstinputlisting[firstline=82,lastline=107]{hw6p1.lst}

\item %d
{\it What are the sample mean and lsmean values for Agent A, and why are they
so different?}

From the SAS output in part (c), the sample mean for Agent A is
\(\bar{y}_{1\cdot}=18.7417\text{ mg}\) and the least squares mean for Agent
A is \(\widehat{\mu}_1=20.0135\text{ mg}\). The sample mean is low because the
Agent A appears in blocks with low response values. The least squares mean
adjusts for the blocks.

\item %e
{\it Find estimates of the agent effects assuming the sum of the effects
equals 0.}

Under the sum-to-zero constraint, the estimated effects are
\(\widehat{\tau}_1=0.8906\), \(\widehat{\tau}_2=0.9656\),
\(\widehat{\tau}_3=-0.8281\), and \(\widehat{\tau}_4=-1.0281\).

\lstinputlisting[firstline=111,lastline=123]{hw6p1.lst}

\end{enumerate}

\pagebreak
\item %2
{\it Factorial experiment for effects of sulphur and nitrogen on red
clover growth. Analyze the data and assess the validity of the model
assumptions. Include a Bonferroni multiple comparison test comparing the 8
cell means.}

\textbf{Model and \(F\)-Test}

The interaction model is
\begin{equation*}
y_{ijk}=\mu+\alpha_i+\beta_j+(\alpha\beta)_{ij}+\epsilon_{ijk}\qquad
\epsilon_{ijk}\overset{iid}{\sim}\mathrm{N}(0\sigma^2)
\end{equation*}
where \(i=1,2\) indexes the nitrogen levels, \(j=1,2,3,4\) indexes the
level of sulfur, and \(k=1,2,3\) indexes the replicates in each combination
of nitrogen and sulfur.

We start by testing for interaction effects,
\begin{align*}
H_0\text{: }&(\alpha\beta)_{11}=(\alpha\beta)_{12}=\dots
=(\alpha\beta)_{24}\text{,} \\
H_1\text{: }&\text{some }(\alpha\beta)_{ij}\neq(\alpha\beta)_{i'j'}\text{.}
\end{align*}
The ANOVA table shows a test statistic of \(F=349.78\) with
\(\text{p-value}<0.0001\). At \(\alpha=0.05\) there is enough evidence to
conclude that there are interaction effects and that the effect of nitrogen on
clover growth depends upon the sulfur level.

\lstinputlisting[firstline=22,lastline=47]{hw6p2glm.lst}

\pagebreak
\textbf{Bonferroni's MCP}

Since the ANOVA \(F\)-test provides evidence of interactions, it is of
interest to perform pariwise comparisons of the mean growth for all pairs of
nitrogren-sulphur combinations,
\begin{align*}
H_0\text{: }&\mu_{ij}=\mu_{i'j'}\text{,} \\
H_1\text{: }&\mu_{ij}\neq\mu_{i'j'}\text{.}
\end{align*}
SAS output with Bonferroni-adjusted p-values appears below. At
\(\alpha=0.05\), we find significant differences between the means of all
pairs except
\begin{itemize}
\item 0 lbs/acre nitrogen, 0 lbs/acre sulfur vs
0 lbs/acre nitrogen, 3 lbs/acre sulfur,
\item  0 lbs/acre nitrogen, 9 lbs/acre sulfur vs
20 lbs/acre nitrogen, 0 lbs/acre sulfur,
\item 0 lbs/acre nitrogen, 9 lbs/acre sulfur vs
20 lbs/acre nitrogen, 6 lbs/acre sulfur, and
\item 20 lbs/acre nitrogen, 0 lbs/acre sulfur vs
20 lbs/acre nitrogen, 6 lbs/acre sulfur.
\end{itemize}

\lstinputlisting[firstline=95,lastline=128]{hw6p2glm.lst}

\pagebreak
\textbf{Model Assumptions and Cramer-Von Mises Test}

The plot of studentized residuals against their fitted values shows no
patterns indicating non-homogeneous variance. However, the Normal Q-Q plot
and the histogram of residuals suggest violations of the Normality assumption.
A test of Normailty will be performed on the residuals.

\includegraphics[width=0.33\linewidth]{clover1}
\includegraphics[width=0.33\linewidth]{clover3}
\includegraphics[width=0.33\linewidth]{clover6}

The Cramer-Von Mises Goodness of Fit Test tests the hypotheses
\begin{align*}
H_0\text{: }&\text{The residuals follow a Normal CDF,} \\
H_1\text{: }&\text{The residuals do not follow a Normal CDF.}
\end{align*}
Output from PROC UNIVARIATE appears below. The Cramer-Von Mises statistic
is \(W^2=0.1319\) with \(\text{p-value}=0.0402\). At \(\alpha=0.05\), we
reject \(H_0\) and conclude that the distibutions of the residuals is
significantly different from Normal. The Normality assumption is violated
and the ANOVA model is inappropriate for these data.

\lstinputlisting[firstline=4,lastline=16]{hw6p2univ.lst}

\pagebreak
\item %3
{\it Each of the following tables represents the cell means from a balanced
\(2\times 2\) factorial completely randomized design with \(n\) replicates
per cell. For each table tell which of the following summs of squares would
be zero for that table: \(SS_A\), \(SS_B\), \(S_{AB}\).}

For any table,

\begin{itemize}
\item\(SS_A=0\) if \(\bar{y}_{\cdot 1\cdot}=\bar{y}_{\cdot 2\cdot}\),
\item\(SS_B=0\) if \(\bar{y}_{1\cdot\cdot}=\bar{y}_{2\cdot\cdot}\),
\item\(SS_{AB}=0\) if \(\bar{y}_{11}-\bar{y}_{12}=\bar{y}_{21}-\bar{y}_{22}\)
and
\(\bar{y}_{11\cdot}-\bar{y}_{21\cdot}=\bar{y}_{12\cdot}-\bar{y}_{22\cdot}\).
\end{itemize}

\begin{minipage}{0.5\linewidth}
\begin{tabular}{c|c|c|}
\multicolumn{1}{c}{} & \multicolumn{2}{c}{Table 1} \\
\multicolumn{1}{c}{} & \multicolumn{2}{c}{Factor A} \\
\cline{2-3}
Factor & ~1~ & 3 \\
\cline{2-3}
B & 5 & 3 \\
\cline{2-3}
\end{tabular}

\begin{itemize}
\item \(\bar{y}_{\cdot 1\cdot}=3\), \(\bar{y}_{\cdot 2\cdot}=3\)
\(\implies\) \(SS_A=0\)
\item \(\bar{y}_{1\cdot\cdot}=2\), \(\bar{y}_{2\cdot\cdot}=4\)
\(\implies\) \(SS_B\neq 0\)
\item \(\bar{y}_{11\cdot}-\bar{y}_{12\cdot}=-2\),
\(\bar{y}_{21\cdot}-\bar{y}_{22\cdot}=2\),\\
\(\bar{y}_{11\cdot}-\bar{y}_{21\cdot}=-4\),
\(\bar{y}_{12\cdot}-\bar{y}_{22\cdot}=0\) \\
\(\implies\) \(SS_{AB}\neq 0\)
\end{itemize}

\end{minipage}\begin{minipage}{0.5\linewidth}
\begin{tabular}{c|c|c|}
\multicolumn{1}{c}{} & \multicolumn{2}{c}{Table 2} \\
\multicolumn{1}{c}{} & \multicolumn{2}{c}{Factor A} \\
\cline{2-3}
Factor & ~1~ & 3 \\
\cline{2-3}
B & 5 & 7 \\
\cline{2-3}
\end{tabular}

\begin{itemize}
\item \(\bar{y}_{\cdot 1\cdot}=3\), \(\bar{y}_{\cdot 2\cdot}=5\)
\(\implies\) \(SS_A\neq 0\)
\item \(\bar{y}_{1\cdot\cdot}=2\), \(\bar{y}_{2\cdot\cdot}=6\)
\(\implies\) \(SS_B\neq 0\)
\item \(\bar{y}_{11}-\bar{y}_{12}=-2\), \(\bar{y}_{21}-\bar{y}_{22}=-2\),\\
\(\bar{y}_{11}-\bar{y}_{21}=-4\), \(\bar{y}_{12}-\bar{y}_{22}=-4\) \\
\(\implies\) \(SS_{AB}=0\)
\end{itemize}

\end{minipage}

\begin{minipage}{0.5\linewidth}
\begin{tabular}{c|c|c|}
\multicolumn{1}{c}{} & \multicolumn{2}{c}{Table 3} \\
\multicolumn{1}{c}{} & \multicolumn{2}{c}{Factor A} \\
\cline{2-3}
Factor & ~1~ & 3 \\
\cline{2-3}
B & 5 & 5 \\
\cline{2-3}
\end{tabular}

\begin{itemize}
\item \(\bar{y}_{\cdot 1\cdot}=3\), \(\bar{y}_{\cdot 2\cdot}=4\)
\(\implies\) \(SS_A\neq 0\)
\item \(\bar{y}_{1\cdot\cdot}=2\), \(\bar{y}_{2\cdot\cdot}=5\)
\(\implies\) \(SS_B\neq 0\)
\item \(\bar{y}_{11}-\bar{y}_{12}=-2\), \(\bar{y}_{21}-\bar{y}_{22}=0\),\\
\(\bar{y}_{11}-\bar{y}_{21}=-4\), \(\bar{y}_{12}-\bar{y}_{22}=-2\) \\
\(\implies\) \(SS_{AB}\neq 0\)
\end{itemize}

\end{minipage}\begin{minipage}{0.5\linewidth}
\begin{tabular}{c|c|c|}
\multicolumn{1}{c}{} & \multicolumn{2}{c}{Table 4} \\
\multicolumn{1}{c}{} & \multicolumn{2}{c}{Factor A} \\
\cline{2-3}
Factor & ~5~ & 3 \\
\cline{2-3}
B & 5 & 3 \\
\cline{2-3}
\end{tabular}

\begin{itemize}
\item \(\bar{y}_{\cdot 1\cdot}=5\), \(\bar{y}_{\cdot 2\cdot}=3\)
\(\implies\) \(SS_A\neq 0\)
\item \(\bar{y}_{1\cdot\cdot}=4\), \(\bar{y}_{2\cdot\cdot}=4\)
\(\implies\) \(SS_B=0\)
\item \(\bar{y}_{11}-\bar{y}_{12}=2\), \(\bar{y}_{21}-\bar{y}_{22}=2\),\\
\(\bar{y}_{11}-\bar{y}_{21}=0\), \(\bar{y}_{12}-\bar{y}_{22}=0\) \\
\(\implies\) \(SS_{AB}=0\)
\end{itemize}

\end{minipage}

\end{enumerate}

\pagebreak
\subsection*{Appendix: SAS Code}

\subsubsection*{Problem 1}

\lstinputlisting[language=SAS]{BIBD_HO.SAS}

\pagebreak
\subsubsection*{Problem 2}

\footnotesize\lstinputlisting[language=SAS]{CLOVR_H2.SAS}

\end{document}
