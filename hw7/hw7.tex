\documentclass[11pt]{article}
\usepackage{fullpage}
\usepackage{graphicx}
\usepackage{float}
\usepackage{amsmath}

\usepackage{listings}
\lstloadlanguages{SAS}
\lstset{basicstyle=\footnotesize\ttfamily,columns=fixed,showstringspaces=false,
  showspaces=false,showtabs=false}

\usepackage{enumitem}
\setlist{parsep=5.5pt}

\usepackage{fancyhdr}
\pagestyle{fancy}
\lhead{Stat 541 Homework \#7}
\chead{April 18, 2016}
\rhead{Kenny Flagg}
\setlength{\headheight}{18pt}
\setlength{\headsep}{2pt}

\title{Stat 541 Homework \#7}
\author{Kenny Flagg}
\date{April 18, 2016}

\begin{document}
\thispagestyle{plain}
\maketitle

\begin{enumerate}

\item %1
{\it Problem 13.1, pages 601. Also provide a practical interpretation of the
variance component estimates in the context of the study.}

%\begin{enumerate}

%\item %a
%{\it Analyze the data from this experiment.}

Since the operators and parts were randomly selected, random effects should
be used. The appropriate model is
\begin{align*}
y_{ijk}&=\mu+\tau_i+\beta_j+(\tau\beta)_{ij}+\epsilon_{ijk}\text{;}\\
\tau_i&\overset{iid}{\sim}\mathrm{N}\left(0,\sigma^2_\tau\right)\text{,}\\
\beta_j&\overset{iid}{\sim}\mathrm{N}\left(0,\sigma^2_\beta\right)\text{,}\\
(\tau\beta)_{ij}&\overset{iid}{\sim}
\mathrm{N}\left(0,\sigma^2_{\tau\beta}\right)\text{,}\\
\epsilon_{ijk}&\overset{iid}{\sim}\mathrm{N}\left(0,\sigma^2\right)
\end{align*}
where \(\tau_i\) is the operator effect, \(\beta_j\) is the part effect, and
\((\tau\beta)_{ij}\) is the operator by part interaction.

The random effects ANOVA output appears on the next page. We first test for
the interaction effect with the hypotheses
\begin{align*}
H_0\text{: }&\sigma^2_{\tau\beta}=0\text{,}\\
H_1\text{: }&\sigma^2_{\tau\beta}>0\text{.}
\end{align*}
The exact \(F\)-statistic is \(F=0.40\) with \(\text{p-value}=0.9270\). There
is little to no evidence of an interaction effect.

Now we test the main effects. The hypotheses for the operator effect are
\begin{align*}
H_0\text{: }&\sigma^2_{\tau}=0\text{,}\\
H_1\text{: }&\sigma^2_{\tau}>0\text{.}
\end{align*}
The exact \(F\)-test has test statistic \(F=0.69\) with
\(\text{p-value}=0.4269\) so there is little evidence that the mean response
varies among operators.

For the part effect, the hypotheses are
\begin{align*}
H_0\text{: }&\sigma^2_{\tau}=0\text{,}\\
H_1\text{: }&\sigma^2_{\tau}>0\text{.}
\end{align*}
The exact \(F\)-statistic is \(F=18.28\) with \(\text{p-value}<0.0001\). There
is strong evidence that the mean measurements vary among parts.

\begin{figure}[H]\centering
\includegraphics[page=1, trim={2in 7.2in 2in 0.25in}, clip]{hw7prob1a}
\end{figure}

%\item %b
%{\it Estimate the variance components using the ANOVA method.}

The variance estmates for the operator and interaction effects are
\(\widehat{\sigma}^2_\tau=0\) and \(\widehat{\sigma}^2_{\tau\beta}=0\). On
average, the measurements do not differ from one operator to another. The
estimated variance for the part effect is \(\widehat{\sigma}^2_\beta=1.614\),
which is small relative to the observed response values and may be acceptably
low. However, the estimated residual variance is \(\widehat{\sigma}^2=1.317\),
so the variability among measurements on the same part is almost as large as
the variability among parts. This implies that the measurement equipment is
not precise enough to measure these parts adequately.

The residual plot shows no concerning patterns and the Normal Q-Q plot shows
little deviation from the diagonal line, so we can assume constant variance
and Normality of the residuals.

\begin{figure}[H]\centering
\includegraphics[page=1, trim={2in 8.5in 2in 0.25in}, clip]{hw7prob1b}
\end{figure}

The residual plot shows no concerning patterns and the Normal Q-Q plot shows
little deviation from the diagonal line, so we can assume constant variance
and Normality of the residuals.

%\begin{figure}[H]
\includegraphics[width=0.5\linewidth]{hw7p1a}
\includegraphics[width=0.5\linewidth]{hw7p1a3}
%\end{figure}

%\end{enumerate}

\item %2
{\it Reconsider Problem 13.1. Consider two scenarios:}
\begin{enumerate}[label=(\roman*)]
\item{\it Suppose that when each measurement was taken, each operator knew the
part number being measured and also recorded the data him/herself.}
\item{\it Suppose that when each measurement was taken, each operator were
``blinded'' to the part number being measured, and a third party recorded
the measurements.}
\end{enumerate}
{\it How might this impact the data analysis?}

In situation (i), measurements on the same part would not be independent.
The observed values could be biased if, for example, an operator remembered
previous measurements of the part and adjusted the values to agree with
previous readings. In situation (ii), all observations would be made
independently and the model used in problem 1 would be appropriate.

\item %3
{\it Suppose the two operators in Problem 13.1 were the only two available
(that is, Operators represents a fixed effect). Which model effects (if any)
now have different \(F\)-test statistics? And, if an effect's \(F\)-test is
different, what would be used for the denominator of the \(F\)-statistic?}

In this case, none of the \(F\)-tests would change. The denominator of the
test for the interaction would still be \(MS_E\) and the denominator for the
tests of the main effects would still be \(MS_{AB}\).

\begin{figure}[H]\centering
\includegraphics[page=1, trim={1in 9.05in 1in 0.25in}, clip]{hw7prob3}
\end{figure}

\item %4
{\it Consider the information on pages 192--194 for Example 5.1 for the battery
design experiment. Suppose the researcher is planning another \(3\times 3\)
factorial experiment. The goal is to determine the sample size required so
that the power of each \(F\)-test (for Material Type, Temperature, and their
interaction) is \(\geq 0.90\).}

\begin{enumerate}

\item %a
{\it Based on this experiment, what is the estimate of \(\sigma\)?}

The estimate is \(\sqrt{MS_E}=25.9848\).

\item %b
{\it Use SAS to determine the desired sample size assuming you use the sample
mean \(\bar{y}_{ij}\) as an estimate of \(\mu_{ij}\) for \(i=1,2,3\) and
\(j=1,2,3\). You can assume \(\alpha=0.05\).}

Four replicates are needed to achieve a power of at least 0.90 for testing the
material effect. Only two replicates are needed for testing the temperature
effect with a power of at least 0.90. For testing the interaction, five
replicates are needed. Thus we need at least five replicates for a total sample
size of 45 in order to obtain a power of 0.90 for all the \(F\)-tests.

\begin{figure}[H]\centering
\includegraphics[page=1, trim={1in 7.05in 1in 0.25in}, clip]{hw7prob4}
\end{figure}

\end{enumerate}

\pagebreak
\item %5
{\it Factors A and B are fixed factors both having 3 fixed levels, and factors
C and D are random factors each having 5 randomly-selected levels. Suppose
\(n=4\) replicates are taken for each A*B*C*D combination. In the Proc GLM
model statement, \(A|B|C|D\) will generate the full four-factor factorial
model.}

\begin{enumerate}

\item %a
{\it Assuming a completely randomized design, generate a table of expected mean
squares for this four-factor factorial experiment.}

\begin{figure}[H]\centering
\includegraphics[width=\linewidth,page=3,trim={0 5.9in 0 0.2in},clip]{hw7prob5}
\end{figure}

\item %b
{\it For what effects are there exact \(F\)-tests? Provide the \(F\)-statistic
in terms of mean squares for these exact tests.}

The following are exact \(F\)-tests:

\begin{table}[H]\centering\renewcommand{\arraystretch}{1.5}
\begin{tabular}{|l|l|}
\hline
Source & \(F\)-statistic \\
\hline
A*B*C & \(F=\frac{MS_{ABC}}{MS_{ABCD}}\) \\
\hline
A*B*D & \(F=\frac{MS_{ABD}}{MS_{ABCD}}\) \\
\hline
A*C*D & \(F=\frac{MS_{ACD}}{MS_{ABCD}}\) \\
\hline
B*C*D & \(F=\frac{MS_{BCD}}{MS_{ABCD}}\) \\
\hline
A*B*C*D & \(F=\frac{MS_{ABCD}}{MS_E}\) \\
\hline
\end{tabular}
\end{table}

\pagebreak
\item %c
{\it For what effects are there approximate \(F\)-tests? Provide the
\(F\)-statistic in terms of mean squares for these approximate tests.}

SAS used the following linear combinations for the approximate \(F\)-tests:

\begin{table}[H]\centering\renewcommand{\arraystretch}{1.5}
\begin{tabular}{|l|l|}
\hline
Source & \(F\)-statistic \\
\hline
A & \(F=\frac{MS_{A}}{MS_{AC}+MS_{AD}-MS_{ACD}}\) \\
\hline
B & \(F=\frac{MS_{B}}{MS_{BC}+MS_{BD}-MS_{BCD}}\) \\
\hline
A*B & \(F=\frac{MS_{AB}}{MS_{ABC}+MS_{ABD}-MS_{ABCD}}\) \\
\hline
C & \(F=\frac{MS_{C}}
{MS_{AC}+MS_{BC}-MS_{ABC}+MS_{CD}-MS_{ACD}-MS_{BCD}+MS_{ABCD}}\) \\
\hline
A*C & \(F=\frac{MS_{AC}}{MS_{ABC}+MS_{ACD}-MS_{ABCD}}\) \\
\hline
B*C & \(F=\frac{MS_{BC}}{MS_{ABC}+MS_{BCD}-MS_{ABCD}}\) \\
\hline
D & \(F=\frac{MS_{D}}
{MS_{AD}+MS_{BD}-MS_{ABD}+MS_{CD}-MS_{ACD}-MS_{BCD}+MS_{ABCD}}\) \\
\hline
A*D & \(F=\frac{MS_{AD}}{MS_{ABD}+MS_{ACD}-MS_{ABCD}}\) \\
\hline
B*D & \(F=\frac{MS_{BD}}{MS_{ABD}+MS_{BCD}-MS_{ABCD}}\) \\
\hline
C*D & \(F=\frac{MS_{CD}}{MS_{ACD}+MS_{BCD}-MS_{ABCD}}\) \\
\hline
\end{tabular}
\end{table}

\item %d
{\it Suppose the four-factor interaction is removed from the model. Answer
parts (b) and (c) again.}


\begin{figure}[H]\centering
\includegraphics[width=\linewidth,page=8,trim={0 6.1in 0 0.25in},clip]{hw7prob5}
\end{figure}

\pagebreak
When the four-factor interaction is removed, these are exact \(F\)-tests:

\begin{table}[H]\centering\renewcommand{\arraystretch}{1.5}
\begin{tabular}{|l|l|}
\hline
Source & \(F\)-statistic \\
\hline
A*B*C & \(F=\frac{MS_{ABC}}{MS_{E}}\) \\
\hline
A*B*D & \(F=\frac{MS_{ABD}}{MS_{E}}\) \\
\hline
A*C*D & \(F=\frac{MS_{ACD}}{MS_{E}}\) \\
\hline
B*C*D & \(F=\frac{MS_{BCD}}{MS_{E}}\) \\
\hline
\end{tabular}
\end{table}

These are the approximate \(F\)-tests performed by SAS:

\begin{table}[H]\centering\renewcommand{\arraystretch}{1.5}
\begin{tabular}{|l|l|}
\hline
Source & \(F\)-statistic \\
\hline
A & \(F=\frac{MS_{A}}{MS_{AC}+MS_{AD}-MS_{ACD}}\) \\
\hline
B & \(F=\frac{MS_{B}}{MS_{BC}+MS_{BD}-MS_{BCD}}\) \\
\hline
A*B & \(F=\frac{MS_{AB}}{MS_{ABC}+MS_{ABD}-MS_{E}}\) \\
\hline
C & \(F=\frac{MS_{C}}
{MS_{AC}+MS_{BC}-MS_{ABC}+MS_{CD}-MS_{ACD}-MS_{BCD}+MS_{E}}\) \\
\hline
A*C & \(F=\frac{MS_{AC}}{MS_{ABC}+MS_{ACD}-MS_{E}}\) \\
\hline
B*C & \(F=\frac{MS_{BC}}{MS_{ABC}+MS_{BCD}-MS_{E}}\) \\
\hline
D & \(F=\frac{MS_{D}}
{MS_{AD}+MS_{BD}-MS_{ABD}+MS_{CD}-MS_{ACD}-MS_{BCD}+MS_{E}}\) \\
\hline
A*D & \(F=\frac{MS_{AD}}{MS_{ABD}+MS_{ACD}-MS_{E}}\) \\
\hline
B*D & \(F=\frac{MS_{BD}}{MS_{ABD}+MS_{BCD}-MS_{E}}\) \\
\hline
C*D & \(F=\frac{MS_{CD}}{MS_{ACD}+MS_{BCD}-MS_{E}}\) \\
\hline
\end{tabular}
\end{table}

\pagebreak
\item %e
{\it Suppose the four-factor and all three-factor interactions are removed
from the model. Answer parts (b) and (c) again.}

\begin{figure}[H]
\includegraphics[page=13,trim={1in 7.2in 1in 0.2in},clip]{hw7prob5}
\end{figure}

These tests are exact:

\begin{table}[H]\centering\renewcommand{\arraystretch}{1.5}
\begin{tabular}{|l|l|}
\hline
Source & \(F\)-statistic \\
\hline
A*B & \(F=\frac{MS_{AB}}{MS_{E}}\) \\
\hline
A*C & \(F=\frac{MS_{AC}}{MS_{E}}\) \\
\hline
B*C & \(F=\frac{MS_{BC}}{MS_{E}}\) \\
\hline
A*D & \(F=\frac{MS_{AD}}{MS_{E}}\) \\
\hline
B*D & \(F=\frac{MS_{BD}}{MS_{E}}\) \\
\hline
C*D & \(F=\frac{MS_{CD}}{MS_{E}}\) \\
\hline
\end{tabular}
\end{table}

\pagebreak
These are approximate \(F\)-tests:

\begin{table}[H]\centering\renewcommand{\arraystretch}{1.5}
\begin{tabular}{|l|l|}
\hline
Source & \(F\)-statistic \\
\hline
A & \(F=\frac{MS_{A}}{MS_{AC}+MS_{AD}-MS_{E}}\) \\
\hline
B & \(F=\frac{MS_{B}}{MS_{BC}+MS_{BD}-MS_{E}}\) \\
\hline
C & \(F=\frac{MS_{C}}
{MS_{AC}+MS_{BC}+MS_{CD}-2MS_{E}}\) \\
\hline
D & \(F=\frac{MS_{D}}
{MS_{AD}+MS_{BD}+MS_{CD}-2MS_{E}}\) \\
\hline
\end{tabular}
\end{table}

\end{enumerate}

\item %6
{\it See Problem 13.21, page 603.}

\begin{enumerate}

\item %a
{\it Answer the second question only: ``If the three-factor and
\((\tau\beta)_{ij}\) interactions do not exist, can all remaining effects be
tested?'' That is, for each model effect, do we have an exact \(F\)-test or do
we have to consider an approximate \(F\)-test?}

The expected mean squares are:

\begin{table}[h]\centering
\begin{tabular}{|l|l|}
\hline
Source & Expected Mean Square \\
\hline
A & \(\sigma^2+b\sigma^2_{\tau\gamma}+bc\sigma^2_{\tau}\) \\
\hline
B & \(\sigma^2+a\sigma^2_{\beta\gamma}+ac\sigma^2_{\beta}\) \\
\hline
C & \(\sigma^2+a\sigma^2_{\beta\gamma}
+b\sigma^2_{\tau\gamma}+ab\sigma^2_{\gamma}\) \\
\hline
AC & \(\sigma^2+b\sigma^2_{\tau\gamma}\) \\
\hline
BC & \(\sigma^2+a\sigma^2_{\beta\gamma}\) \\
\hline
\end{tabular}
\end{table}

There are exact \(F\)-tests for factor A (denominator is \(MS_{AC}\)),
factor B (denominator is \(MS_{BC}\)), and the AC and BC interactions
(denominator is \(MS_{E}\)).

\item %b
{\it For each effect having an associated approximate \(F\)-test, provide a
linear combination of mean squares appearing in the numerator and in the
denominator of the \(F\)-statistic.}

The only approximate \(F\)-test is the test for the factor C effect. One
possible \(F\)-statistic for this test is:
\begin{align*}
F=\frac{MS_{C}+MS_{E}}{MS_{AC}+MS_{BC}}
\end{align*}

\end{enumerate}

\end{enumerate}

\pagebreak
\subsection*{Appendix: SAS Code}

\subsubsection*{Problems 1--3}

\footnotesize\lstinputlisting[language=SAS]{p13_1.sas}

\subsection*{Problem 4}

\footnotesize\lstinputlisting[language=SAS]{prob4.sas}

\pagebreak
\subsection*{Problem 5}

\footnotesize\lstinputlisting[language=SAS]{prob5.sas}

\pagebreak
\subsection*{Problem 6}

\footnotesize\lstinputlisting[language=SAS]{prob6.sas}

\end{document}

