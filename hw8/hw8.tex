\documentclass[11pt]{article}
\usepackage{fullpage}
\usepackage{graphicx}
\usepackage{float}
\usepackage{amsmath}

\usepackage{listings}
\lstloadlanguages{SAS}
\lstset{basicstyle=\footnotesize\ttfamily,columns=fixed,showstringspaces=false,
  showspaces=false,showtabs=false}

\usepackage{enumitem}
\setlist{parsep=5.5pt}

\usepackage{fancyhdr}
\pagestyle{fancy}
\lhead{Stat 541 Homework \#8}
\chead{April 25, 2016}
\rhead{Kenny Flagg}
\setlength{\headheight}{18pt}
\setlength{\headsep}{2pt}

\title{Stat 541 Homework \#8}
\author{Kenny Flagg}
\date{April 25, 2016}

\begin{document}
\thispagestyle{plain}
\maketitle

\begin{enumerate}

\item %1
{\it Problem 14.4 (pages 637) but suppose, however, that five operators were
randomly selected for each of the six jobs. Be sure to state the model and
hypotheses to be tested, check the model assumptions, and provide estimates of
all variance components. You can assume that Operator is also a random effect.}

The appropriate model is
\begin{align*}
y_{ijk}&=\mu+\tau_{i}+\beta_{j(i)}+\epsilon_{k(ij)};\\
\tau_{i}&\overset{iid}{\sim}\mathrm{N}(0,\sigma^2_\tau),\\
\beta_{j}&\overset{iid}{\sim}\mathrm{N}(0,\sigma^2_\beta),\\
\epsilon_{k}&\overset{iid}{\sim}\mathrm{N}(0,\sigma^2)
\end{align*}
where \(\tau_{i}\) is the job effec and \(\beta_{j}\) is the operator effect
within job.

The hypotheses of interest are \(H_0\text{: }\sigma^2_\tau=0\) and
\(H_1\text{: }\sigma^2_\tau>0\). With a test statistic of \(F=31.47\) and
\(\text{p-value}<0.0001\), we have suffiecient evidence to reject \(H_0\) at
a 0.05 significance level. There is strong evidence that the mean time varies
between jobs, so a common standard should not be used for all jobs.

\begin{figure}[H]\centering
\includegraphics[page=12,trim={2in 7.45in 2in 0.3in},clip]{hw8prob1.pdf}
\end{figure}

\begin{figure}[H]\centering
\includegraphics[page=15,trim={2in 4.3in 2in 5.3in},clip]{hw8prob1.pdf}
\end{figure}

The variance estimates are \(\sigma^2_\tau=4.31\), \(\sigma^2_\beta=0.215\)
and \(\sigma^2=0.954\). The job is by far the largest component of the
variability in completion time.

\begin{figure}[H]\centering
\includegraphics[width=0.4\linewidth]{hw8p1.png}
\includegraphics[width=0.4\linewidth]{hw8p19.png}\\
\includegraphics[width=0.4\linewidth]{hw8p13.png}
\includegraphics[width=0.4\linewidth]{hw8p16.png}
\end{figure}

Some observations with large predicted values also have large residuals, but
there is not a strong increasing trend. The boxplots for individual operators
show no signs of a serious heterogeneity of variance problem. The Q-Q plot
shows some deviation from the diagonal line near the center, but it is not
surprising to see this sort of pattern with only two replicates for each
operator. The histogram of residuals looks reasonably bell-shaped, so it is
reasonable to assume Normality.

\pagebreak
\item %2
{\it Two-stage nested design with 12 machines, 4 spindles nested within each
machine, and 5 replicates within each spindle}

\begin{enumerate}

\item %a
{\it Fill in the d.f. column assuming all 240 values were collected.}

\begin{table}[H]\centering
\begin{tabular}{lr}
Source & d.f. \\
\hline
Machine & 11 \\
Spindle(Machine) & 36 \\
Error & 192 \\
\hline
Total & 239
\end{tabular}
\end{table}

\item %b
{\it Suppose the three \(\mathbf{X}\) values were missing for Spindle 3. Fill
in the partial ANOVA table again.}

\begin{table}[H]\centering
\begin{tabular}{lr}
Source & d.f. \\
\hline
Machine & 11 \\
Spindle(Machine) & 36 \\
Error & 189 \\
\hline
Total & 236
\end{tabular}
\end{table}

\item %c
{\it Suppose the five \(\mathbf{Y}\) values for Spindle 3 from Machine 12 are
missing. Fill in the partial ANOVA table again.}

\begin{table}[H]\centering
\begin{tabular}{lr}
Source & d.f. \\
\hline
Machine & 11 \\
Spindle(Machine) & 35 \\
Error & 188 \\
\hline
Total & 234
\end{tabular}
\end{table}

\item %d
{\it Suppose the the three \(\mathbf{X}\) values and the five \(\mathrm{Y}\)
values were missing. Fill in the partial ANOVA table again.}

\begin{table}[H]\centering
\begin{tabular}{lr}
Source & d.f. \\
\hline
Machine & 11 \\
Spindle(Machine) & 35 \\
Error & 185 \\
\hline
Total & 231
\end{tabular}
\end{table}

\end{enumerate}

\pagebreak
\item %3
{\it Using the data in Problem 14.3 (page 637), what are the model matrix
\(\mathbf{X}\) and the corresponding vector \(y\) assuming
\(\displaystyle\sum_{i=1}^3\alpha_i=0\) and
\(\displaystyle\sum_{j(i)=1}^2\beta_j=0\) for \(i=1,2,3\)? Write \(\mathbf{X}\)
in so that there is a column for each model parameter.}

\begin{equation*}
\mathbf{X}=\begin{bmatrix}
1 & 1 & 0 & 0 & 1 & 0 & 0 & 0 & 0 & 0 \\
1 & 1 & 0 & 0 & 1 & 0 & 0 & 0 & 0 & 0 \\
1 & 1 & 0 & 0 & 1 & 0 & 0 & 0 & 0 & 0 \\
1 & 1 & 0 & 0 & 1 & 0 & 0 & 0 & 0 & 0 \\

1 & 1 & 0 & 0 & 0 & 1 & 0 & 0 & 0 & 0 \\
1 & 1 & 0 & 0 & 0 & 1 & 0 & 0 & 0 & 0 \\
1 & 1 & 0 & 0 & 0 & 1 & 0 & 0 & 0 & 0 \\
1 & 1 & 0 & 0 & 0 & 1 & 0 & 0 & 0 & 0 \\

1 & 0 & 1 & 0 & 0 & 0 & 1 & 0 & 0 & 0 \\
1 & 0 & 1 & 0 & 0 & 0 & 1 & 0 & 0 & 0 \\
1 & 0 & 1 & 0 & 0 & 0 & 1 & 0 & 0 & 0 \\
1 & 0 & 1 & 0 & 0 & 0 & 1 & 0 & 0 & 0 \\

1 & 0 & 1 & 0 & 0 & 0 & 0 & 1 & 0 & 0 \\
1 & 0 & 1 & 0 & 0 & 0 & 0 & 1 & 0 & 0 \\
1 & 0 & 1 & 0 & 0 & 0 & 0 & 1 & 0 & 0 \\
1 & 0 & 1 & 0 & 0 & 0 & 0 & 1 & 0 & 0 \\

1 & 0 & 0 & 1 & 0 & 0 & 0 & 0 & 1 & 0 \\
1 & 0 & 0 & 1 & 0 & 0 & 0 & 0 & 1 & 0 \\
1 & 0 & 0 & 1 & 0 & 0 & 0 & 0 & 1 & 0 \\
1 & 0 & 0 & 1 & 0 & 0 & 0 & 0 & 1 & 0 \\

1 & 0 & 0 & 1 & 0 & 0 & 0 & 0 & 0 & 1 \\
1 & 0 & 0 & 1 & 0 & 0 & 0 & 0 & 0 & 1 \\
1 & 0 & 0 & 1 & 0 & 0 & 0 & 0 & 0 & 1 \\
1 & 0 & 0 & 1 & 0 & 0 & 0 & 0 & 0 & 1 \\

0 & 1 & 1 & 1 & 0 & 0 & 0 & 0 & 0 & 0 \\
0 & 0 & 0 & 0 & 1 & 1 & 0 & 0 & 0 & 0 \\
0 & 0 & 0 & 0 & 0 & 0 & 1 & 1 & 0 & 0 \\
0 & 0 & 0 & 0 & 0 & 0 & 0 & 0 & 1 & 1 \\
\end{bmatrix}\qquad y=\begin{bmatrix}
12 \\  9 \\ 11 \\ 12 \\
 8 \\  9 \\ 10 \\  8 \\
14 \\ 15 \\ 13 \\ 14 \\
12 \\ 10 \\ 11 \\ 13 \\
14 \\ 10 \\ 12 \\ 11 \\
16 \\ 15 \\ 15 \\ 14 \\

0 \\ 0 \\ 0 \\ 0
\end{bmatrix}
\end{equation*}

\pagebreak
\item %4
{\it Redo the previous problem with the reduced column form for
 \(\mathbf{X}\).}

\begin{equation*}
\mathbf{X}=\begin{bmatrix}
1 & 1 & 0 & 1 & 0 & 0 \\
1 & 1 & 0 & 1 & 0 & 0 \\
1 & 1 & 0 & 1 & 0 & 0 \\
1 & 1 & 0 & 1 & 0 & 0 \\

1 & 1 & 0 & -1& 0 & 0 \\
1 & 1 & 0 & -1& 0 & 0 \\
1 & 1 & 0 & -1& 0 & 0 \\
1 & 1 & 0 & -1& 0 & 0 \\

1 & 0 & 1 & 0 & 1 & 0 \\
1 & 0 & 1 & 0 & 1 & 0 \\
1 & 0 & 1 & 0 & 1 & 0 \\
1 & 0 & 1 & 0 & 1 & 0 \\

1 & 0 & 1 & 0 & -1& 0 \\
1 & 0 & 1 & 0 & -1& 0 \\
1 & 0 & 1 & 0 & -1& 0 \\
1 & 0 & 1 & 0 & -1& 0 \\

1 & -1& -1& 0 & 0 & 1 \\
1 & -1& -1& 0 & 0 & 1 \\
1 & -1& -1& 0 & 0 & 1 \\
1 & -1& -1& 0 & 0 & 1 \\

1 & -1& -1& 0 & 0 & -1 \\
1 & -1& -1& 0 & 0 & -1 \\
1 & -1& -1& 0 & 0 & -1 \\
1 & -1& -1& 0 & 0 & -1
\end{bmatrix}\qquad y=\begin{bmatrix}
12 \\  9 \\ 11 \\ 12 \\
 8 \\  9 \\ 10 \\  8 \\
14 \\ 15 \\ 13 \\ 14 \\
12 \\ 10 \\ 11 \\ 13 \\
14 \\ 10 \\ 12 \\ 11 \\
16 \\ 15 \\ 15 \\ 14
\end{bmatrix}
\end{equation*}

\end{enumerate}

\pagebreak
\subsection*{Appendix: SAS Code}

\subsubsection*{Problems 1}

\footnotesize\lstinputlisting[language=SAS]{p14_4ho3.sas}

\end{document}

